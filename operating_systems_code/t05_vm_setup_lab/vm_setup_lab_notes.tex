\chapter{VM setup lab}
\label{ch:vm-setup-lab}

Note carefully:
\begin{itemize}
\item \textbf{If you haven't completed your GitLab setup for lab work you need to go back to \autoref{ch:gitlab-setup}.}
\item \textbf{If you haven't completed the XOA setup you need to go back to \autoref{ch:xoa-setup}!}
\end{itemize}
Do not continue without both of the above prerequisities done!

\section{Setting up the virtual machine}

\begin{enumerate}
\item Login to \url{https://xoa.comp.dkit.ie/}
\item On the left click \textbf{VMs}.
\item Click the green \textbf{New VM} button.
\item Drop down to \textit{Primary Pool} (most likely the only option).
\item Under \textbf{Template} choose \textit{Ubuntu Server Quick Instance}.
\item Change the \textbf{Name} to \texttt{grantp\_os\_linux}. \textbf{ALWAYS name EVERY VM you create starting with your student number}. \label{step:xoa-setup-vm-name}
\item Under description you should enter something about the machine.
  Very helpful when you have multiple machines for different modules and projects, e.g. Linux VM for Operating Systems.
\item Under \textbf{Disks} change the \textbf{Name} to be the same as that in step~\ref{step:xoa-setup-vm-name}.
\item Click the \textbf{Create} button.
\item Choose the \textbf{Console} tab and you can see the machine starting up.
\item Login to the machine as username \texttt{administrator} and usual lab password (will be given in class if needed). 
\end{enumerate}

\section{Clone your Lab code repository}

\begin{enumerate}
\item At the Bash prompt type git clone and the link to your GitLab repository:
\begin{verbatim}
git clone https://gitlab.comp.dkit.ie/grantp/os_labs.git
#         ^
#         | obviously replace with your link!
\end{verbatim}
\item List out your home directory using the command
\begin{verbatim}
ls -l
#  ^
#  | the -l switch puts it in detail format
\end{verbatim}
\item Change into the \texttt{os\_labs} folder using the \texttt{cd} command.
\item Make a new folder for this week, \texttt{week02} using the \texttt{mkdir} command.
\item Change into your newly created \texttt{week02} folder.
\item Use the \texttt{lshw} command to print out information about the virtual machine's hardware.
  Because it involves hardware it has to be run as root (UNIX word for administrator) using \texttt{sudo}:
\begin{verbatim}
sudo lshw
\end{verbatim}
\item To document that you got this far, we will capture this command's output to a text file using the shell redirection operator \texttt{>}:
\begin{verbatim}
sudo lshw > hardware.txt
\end{verbatim}
\item Use \texttt{ls} to confirm the \texttt{hardware.txt} file is there.
  (If the name is wrong use \texttt{mv} to rename it.)
\item Add, commit and push as follows:
\begin{verbatim}
git add hardware.txt
git commit -m 'hardware of virtual machine'
git push 
\end{verbatim}
\end{enumerate}
