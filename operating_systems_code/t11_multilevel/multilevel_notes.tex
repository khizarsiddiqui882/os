\chapter{Multi-level feedback}
\label{ch:multi-level-feedback}

% \begin{goals}
%   \begin{enumerate}
%   \item \textbf{Recognise} inherent simplifications in basic scheduling algorithms.
%   \item \textbf{State} key assumptions and problem for real-world scheduling.
%   \item \textbf{Conceptualise} the multi-level feedback queue.
%   \item \textbf{Mitigate} against starvation and gaming in MLFQ systems.
%   \end{enumerate}
% \end{goals}

% \begin{reading}
%   \begin{enumerate}
%   \item
%     \citet[Chapter 8]{arpaci-dusseau:2015:operating}\\
%     \url{http://pages.cs.wisc.edu/~remzi/OSTEP/cpu-sched-mlfq.pdf}
%   \end{enumerate}
% \end{reading}

\section{Real-world scheduling}

So far, the basic schedulers we've looked at have relied on various simplifying assumptions.

\subsection{Aims}

\begin{enumerate}
\item Optimise the turnaround time (like SJF, STCF).
\item Minimise response time (like RR)
\end{enumerate}

\subsection{Problems}

\begin{enumerate}
\item We don't know how long a process will take.
\item We don't know what I/O a process might/might not undertake.
\item We don't know how long any particular I/O operation might be.
\item A repeated I/O operation mightn't take the same time each time.
\end{enumerate}


\section{Multi-level feedback queue (MLFQ)}

Multiple queues, with each assigned a different \textbf{priority} level:
\begin{itemize}
\item A job is always in one single queue.
\end{itemize}

\section{MLFQ basic rules}
\label{sec:mlfq-rules}

\citet[Chapter 8]{arpaci-dusseau:2015:operating} outlines the basic MLFQ rules:

\begin{enumerate}
\item If $P_A>P_B$, job $A$ runs. \label{item:mlfq-rule:greater-priority}
\item If $P_A=P_B$, $A$ and $B$ run RR. \label{item:mldq-rule:equal-priority}
\end{enumerate}

Although we could have priorities known a-priori, MLFQ implementations usually use the scheduler to assign and vary job priorities based on observed behaviour:


\begin{enumerate}
  \setcounter{enumi}{2}
\item Job arrives in highest priority queue. \label{item:mlfq-rule:arrives}
\item While running, if the job:
  \begin{enumerate}
  \item uses entire time, priority is reduced. \label{item:mlfq-rule:priority-down}
  \item gives up before time slice finished, stays the same. \label{item:mlfq-rule:stays-same}
  \end{enumerate}

\end{enumerate}

\section{MLFQ Problems}
\label{sec:mlfq-problems}

\begin{description}
\item[Starvation:] Lower-running process will never get any CPU time if interactive jobs keep appearing.
\item[Changing process:] what happens if a job alternates between batch computation and interactivity?
\item[Gaming:] since a process can control whether it stays in the same queue or is deprioritised, it can issues spurious I/O requests to trick the scheduler.
\end{description}

\section{Priority boost}

We add a further rule: 

\begin{enumerate}
  \setcounter{enumi}{4}
\item After some time period $S$, move all jobs in system to topmost queue \label{item:mlfq-rule:priority-boost}
\end{enumerate}

The value of $S$ is critical:
\begin{itemize}
\item too high $S$ will starve long-running batch jobs
\item too low $S$ will prevent interactive jobs getting share of resources.
\end{itemize}

\section{Accounting}

Our MLFQ scheduler can still be gamed by a process to retain its priority level.
Instead we can say:
\begin{enumerate}
  \setcounter{enumi}{3}
\item Once a job's time allotment at a given level is used up, it is demoted.
\end{enumerate}
With this policy in place, a process can no longer retain its current priority by issuing (possibly bogus / throwaway) I/O requests.

\section{MLFQ parameters}

MLFQ relies on a number of key parameters:

\begin{enumerate}
\item Number of queues
\item Time slice
\item How often to priority boost?
\end{enumerate}

\subsection{Queue-specific time slices}

Most MLFQ systems will use differing time slices on different queues:
\begin{description}
\item[High-priority queues] use shorter time slices (e.g. $<10$~ms).
\item[Lower priority queues] use longer time slices (e.g. $n\times100$~ms).
\end{description}

\subsection{Priority restriction}

\begin{itemize}
\item Some schedulers will prevent certain processes gaining highest priority (e.g. may restrict to OS processes, not user jobs).
\item Can sometimes request specific priority (e.g. the \texttt{nice} utility on UNIX).
\end{itemize}


