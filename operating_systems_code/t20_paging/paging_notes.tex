\chapter{Paging}
\label{ch:paging}

%\begin{goals}
%  \begin{enumerate}
%  \item \textbf{Evaluate} inherent difficulties with segmentation.
%  \item \textbf{Describe} the mechanism of paging to virtualise memory.
%  \end{enumerate}
%\end{goals}

%\begin{reading}
%  \begin{enumerate}
%  \item
%    \citet[Chapter 18]{arpaci-dusseau:2015:operating}\\
%    \url{http://pages.cs.wisc.edu/~remzi/OSTEP/vm-paging.pdf}
%  \end{enumerate}
%\end{reading}

\section{Basic idea}

Paging divides:
\begin{description}
\item[virtual address space] into fixed-size \textbf{pages}, or chunks of data, identified by a virtual page number (VPN).
\item[physical address space] into fixed-size \textbf{page frames}, or chunks of memory, identified by a physical page number (PPN) or physical frame number (PFN).
\end{description}
Each page frame can contain one page of data, or it can be empty.
Note that:
\begin{align}
  \mbox{PFN} & \equiv \mbox{PPN}
\end{align}

The address space of each process is divided into a number of pages.
Each page is then assigned to a page frame.
The OS maintains a per-process page table that maps a virtual page number to a page frame.

\section{Addressing}

To divide the address space into pages, we need a way to map a given virtual address to a page frame in physical memory identified by its physical page number.

Each virtual address is considered to be made up of a VPN and offset.
The VPN is a configurable number of the most significant bits on the address, \autoref{fig:address-vpn-offset}.
The bits that form the VPN are known as the mask.

\begin{figure}[htbp]
  \centering
\begin{verbatim}
    1 1 0 1 1 1 0 0
    --- -----------
     |       |
     |    offset
    VPN
\end{verbatim}
  \caption{Address as a VPN and offset}
  \label{fig:address-vpn-offset}
\end{figure}

The offset is then added to the PPN to give the final physical address as in \autoref{fig:mapping-virtual-physical-paging}. 

\begin{figure}[htbp]
  \centering
  INSERT DIAGRAM HERE!
  \caption{Mapping virtual to physical address using paging}
  \label{fig:mapping-virtual-physical-paging}
\end{figure}

\section{Page table}

The page table is a per-process data structure that contains the VPN-to-PPN mappings. Each page table entry (PTE) contains:

\begin{description}
\item[VPN:] the ``key'' of the mapping
\item[PPN / PFN:] the physical page number or physical frame number that stores this page.
\item[Flags] that give more information about the page: 
  \begin{description}
  \item[Invalid] (or as the inverse, valid) blocks translations where a mapping does not exist. Allows a sparse memory space.
  \item[Protection bits] as with segmentation (read-execute, read-write etc)
  \item[Present] shows whether a page is actually in memory or has been swapped out (coming later!)
  \end{description}
\end{description}

It could be imagined as an array of structs, indexed by the VPN.

\subsection{Implementation}

In Intel x86, the page table is in a standard format in RAM pointed to by the CR3 register.

\section{Evaluation}

\subsection{Compared to segmentation}

\begin{itemize}
\item Avoids external fragmentation.
\item Allows sparse address spaces.
\end{itemize}

\subsection{Issues}

\begin{itemize}
\item Large number of memory accesses to page table.
\item Size of page table in physical memory.
\end{itemize}


\section{Lab exercise}

A computer has an 8-bit virtual address space with a 3-bit VPN.
It has an 8-bit physical address space also with a 3-bit VPN. 

\begin{enumerate}
\item
	Draw the virtual address space layout for a process with 
    1 page each of code, heap and stack. 
	Label the start and end addresses of each valid region.
	
\item
	Assuming that PFNs 0, 1, 3, 7 are already occupied,
    construct page table for the process showing:
	\begin{enumerate}
	\item VPN
	\item Whether valid
	\item PFN
	\item Protection bits (You can use any free PFNs as you see fit.)
	\end{enumerate}

 	
\item
	Convert the decimal virtual addresses 13, 35 to its physical address. 
\end{enumerate}
