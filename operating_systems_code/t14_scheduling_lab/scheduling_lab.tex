\chapter{Scheduling lab}
\label{ch:scheduling-lab}

\section{Setup}

Do do this lab you will need:
\begin{enumerate}
\item Python
\item Bash / PowerShell
\end{enumerate}

\section{Exercises}

This lab is to be completed in \texttt{week05} of your \texttt{os\_labs} GitLab folder.

\begin{enumerate}

\item
  Copy all the \texttt{.json} and \texttt{.py} files from this folder into your \texttt{week05} folder. 

\item 
  You are given a number of jobs in the file \texttt{jobs1.json}.
  Write a python program called \texttt{job\_list.py} that will read in this file and display it in the following format to the screen:
\begin{verbatim}
J0 [arrives t=0, duration 10]
J1 [arrives t=0, duration 10]
J2 [arrives t=0, duration 20]
\end{verbatim}
  
\item
  Modify your \texttt{job\_list.py} program to accept the filename as an argument (NOT via input).
  (Hint: look at \texttt{job\_generator.py} for how to do this using \texttt{argparse} module.)
  Verify that it works with the provided \texttt{jobs2.json} and \texttt{jobs3.json}.

\item
  Use the \texttt{job\_generator.py} program to generate a new list of 7 jobs, max arrival time 0 and max duration 1000 in the file \texttt{jobs4.json}.
  (The \texttt{job\_generator.py} program doesn't directly write to files - use the shell redirection > in bash or the \texttt{Out-File} cmdlet in PowerShell to redirect its output to a file.)

\item
  Test that your \texttt{job\_list.py} program can correctly display the new \texttt{jobs4.json} file that you have made.

\item
  Save the output of your \texttt{job\_list.py} program as \texttt{job\_list\_output.txt} using either > or \texttt{Out-File} as appropriate.

\item \label{step:scheduling-lab:fifo}
  Write a new python program called \texttt{fifo.py} that reads a specified input file and computes for each job in the case of FIFO scheduling:
  \begin{enumerate}
  \item The time it first runs.
  \item The time it completes.
  \item Its response time.
  \item Its turnaround time.
  \end{enumerate}
  Also compute the average response time and average turnaround time. 
  Output should be in the following format, ordered by when the job is started:
\begin{verbatim}
J0 [start = 0, completes = 10, response = 0, turnaround = 10]
J1 [start = 10, completes = 20, response = 10, turnaround = 20]
J2 [start = 20, completes = 30, response = 20, turnaround = 30]
average response time: 15
average turnaround time: 30
\end{verbatim}

\item
  Write a new Python program called \texttt{sjf.py} that does the same as step~\ref{step:scheduling-lab:fifo} but for a SJF scheduler.

\item
  Construct an input file to demonstrate the convoy phenomenon named \texttt{convoy_jobs.json}.
  Demonstrate by capturing the output of \texttt{fifo.py} and \texttt{sjf.py} to a single file \texttt{convoy_output.txt} that the SJF scheduler avoids the convoy phenomenon.
  (Hint: look up how to append to a file!)
  
\end{enumerate}

\section{Submission}

Make sure to add, commit and push your work to your GitLab repository.

