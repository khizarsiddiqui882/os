\chapter{Processes}
\label{ch:processes}

% \begin{goals}
%   \begin{enumerate}
%   \item \textbf{Differentiate} among running, ready and blocked processes.
%   \item \textbf{Describe} the transitions among process states.
%   \end{enumerate}
% \end{goals}

% \begin{reading}
%   \begin{enumerate}
%     \item
%     \citet[Chapter 4]{arpaci-dusseau:2015:operating}\\
%     \url{http://pages.cs.wisc.edu/~remzi/OSTEP/cpu-mechanisms.pdf}
%   \end{enumerate}
% \end{reading}

\section{Programs and processes}

\begin{description}
\item[Program] is a sequence of instructions.\\
  Also called a \textbf{task} or \textbf{activity}
\item[Process] is an instance of a particular program.
\end{description}

It is very important that you don't confuse the terms program and process.
They aren't the same!

\section{Role of the OS}

The OS must:
\begin{enumerate}
\item allocate resources to processes
\item enable processes to share and exchange information,
\item protect the resources of each process from other processes and
\item enable synchronization among processes
\end{enumerate}


\section{System calls}

A process uses what are called System Calls to (primarily) access hardware. 

\begin{itemize}
\item A system call is used by application (user) programs to request service from the operating system.
\item An operating system (i.e. its kernel) can access a system's hardware directly, but a user program is not given direct access to the hardware.
\item This is done so that the kernel can mediate all access to hardware.
\item Program makes a system call to ask the kernel to take some action on its behalf. 
\end{itemize}

\section{Multiprogramming}

\section{Process Control Blocks}

\section{Process creation}

\section{Process terminaton}

\section{2-state model}

\subsection{Process states}

\begin{description}
\item[Running:] process is running currently on the CPU.
\item[Ready:] process could run if the operating system chose to.
%\item[Blocked:] process must wait for an I/O operation to complete.
\end{description}

\begin{figure}[htbp]
  \centering
  \includegraphics[width=1.0\linewidth]{two_state_transition}
  \caption{2-state process model}
  \label{fig:2-state-process-model}
\end{figure}

\subsection{Transition}


\section{5-State process model}

\begin{figure}[htbp]
  \centering
  \includegraphics[width=1.0\linewidth]{state_transitions}
  \caption{5-state process model}
  \label{fig:5-state-process-model}
\end{figure}

\section{7-State process model}

\begin{figure}[htbp]
  \centering
  \includegraphics[width=1.0\linewidth]{7_state_model}
  \caption{7-state process model}
  \label{fig:7-state-process-model}
\end{figure}

