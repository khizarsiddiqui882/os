\chapter{Switching}
\label{ch:switching}

% \begin{goals}
%   \begin{enumerate}
%   \item \textbf{Describe} the typical mechanisms used to time-share among processes.
%   \item \textbf{Differentiate} between cooperative and preemptive process switching.
%   \item \textbf{Explain} how machine state is saved during a context switch.
%   \end{enumerate}
% \end{goals}

% \begin{reading}
%   \begin{enumerate}
%     \item
%     \citet[Chapter 4]{arpaci-dusseau:2015:operating}\\
%     \url{http://pages.cs.wisc.edu/~remzi/OSTEP/cpu-mechanisms.pdf}
%   \end{enumerate}
% \end{reading}

So far, we know that the OS takes control when a trap occurs from a process.
This can be used by the CPU to do the work requested in the system call, but it could also be used by the CPU to run a different process instead.
Here, we take that idea further of itself to show how the CPU may be time-shared amongst multiple processes.

\section{Co-operative multitasking}

Co-operating multitasking is a pattern where a process ``gives up'' or yields control of the CPU to the operating system.

\subsection{Sequence of events}

\begin{enumerate}
\item Program issues a \textit{yield} system call and traps into the kernel.
\item Kernel then has control and can \textit{schedule} a different process.
\item Or, the kernel can resume back into the original process by issuing a return-from-trap instruction.
\end{enumerate}


\subsection{Problem}

Co-operative multitasking depends on programs' behaving according to the rules and regularly issuing the yield instruction.
\begin{itemize}
\item Badly written programs.
\item Inconsiderate programmers.
\end{itemize}


\section{Pre-emptive multitasking}

We need a way for the OS to take control without help from the program.
An opportunity exists with the hardware interrupts, \autoref{sec:interrupts}, provided by the CPU.

\subsection{Setup}

For pre-emptive multitasking to work, the operating system at boot time must:

\begin{enumerate}
\item Tell the hardware what code to run when the interrupt occurs.
\item Switch the timer on so that the interrupts will be generated.
\end{enumerate}

\subsection{Hardware support}

The hardware must provide some basic support for pre-emptive multitasking: the machine state must be saved sufficiently (onto a kernel stack) so that it can be restored. 

\section{Context switching}

\subsection{Decision}

After the OS regains control (either co-operatively or preemptively), it must decide whether to resume the process or switch to a different one.

The scheduler is the part of the operating system that makes this decision.
We will explore it in later classes.

If the operating system decides to switch to a different process, it is said that a context switch takes place.

\subsection{Context switch}

We assume that Process~A is executing and that the kernel is going to switch to Process~B when the timer fires (or yield is called).

\begin{enumerate}
\item Timer interrupt occurs.
\item Hardware saves A's registers to kernel stack for A.
\item Trap into kernel occurs, kernel jumps to trap handler.
\item Kernel handles the trap and decides to run process B.
\item Saves registers of A to proc struct A.
\item Restores registers of B from proc struct B.
\item Switches to kernel stack B.
\item Return from trap issued into B.
\item H/W restores B's registers from kernel stack.
\item Returns from trap (now in the context of B).
\end{enumerate}

