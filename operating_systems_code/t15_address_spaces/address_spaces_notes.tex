\chapter{Address spaces}
\label{ch:address-spaces}

% \begin{goals}
%   \begin{enumerate}
%   \item \textbf{Conceptualise} the idea of an abstracted address space.
%   \item \textbf{Identify} the goals of a virtual memory system.
%   \end{enumerate}
% \end{goals}

% \begin{reading}
%   \begin{enumerate}
%   \item
%     \citet[Chapter 13]{arpaci-dusseau:2015:operating}\\
%     \url{http://pages.cs.wisc.edu/~remzi/OSTEP/vm-intro.pdf}
%   \end{enumerate}
% \end{reading}

\section{Address space}

The physical memory of a computer is divided up into blocks of a specific size.
\begin{itemize}
\item Each block has an address so it can be accessed to read from or write to the block, \autoref{fig:physical-address-space}.
\item Consider a situation of a single-tasking operating system where the operating system occupies the lowest portion and the running process has access to the upper portion.
\end{itemize} 

\begin{figure}[htbp]
  \centering
\begin{verbatim}
                  Address
+---------------+ 0
|               |
| Operating     |
| system        |
|               |
+---------------+ >0, <max
|               |
| User program  |
| instructions  |
| and data      |
|               |
|               |
|               |
|               |
|               |
+---------------+ max
\end{verbatim}
  \caption{Physical address space}
  \label{fig:physical-address-space}
\end{figure}

\section{Multitasking}

We have seen that the CPU is shared in time.
Memory is instead shared in space.

\begin{figure}[htbp]
  \centering
\begin{verbatim}
                  Address
+---------------+ 0
| Operating     |
| system        |
+---------------+
| Process 1     |
|               |
+---------------+
| free space    |
|               |
+---------------+
| Process 2     |
|               |
+---------------+
| Process 3     |
|               |
+---------------+
| free space    |
|               |
+---------------+
| Process 4     |
|               |
+---------------+ max
\end{verbatim}
  \caption{Multi-program address space}
  \label{fig:multi-address-space}
\end{figure}


\section{Process address space}

Focusing on the address space of a single process, there are three key elements:
\begin{description}
\item[Instructions:] machine language instructions for direct execution by the processor.
\item[Stack:] for anything \texttt{push}ed on or \texttt{pop}ped off the stack. In general this means arguments and local variables.
\item[Heap:] for dynamically allocated memory using \texttt{malloc} or \texttt{new}.
\end{description}

\begin{figure}[htbp]
  \centering
\begin{verbatim}
                  Address
+---------------+ 0
| Program code  | <--- IP
| /instructions |
+---------------+ >0, <max
| Heap          |
|               |
+---------------+
| free          |
| space         |
+---------------+ <--- SP
|               |
| Stack         |
|               |
+---------------+ max
\end{verbatim}
  \caption{Process address space}
  \label{fig:process-address-space}
\end{figure}

\section{Virtual memory system}

The basic function of a virtual memory system is to take a memory address from the CPU and translate it to a physical address in memory.

Most virtual memory systems rely on hardware support.

(As an analogy, consider a block of Airbnb apartments: each apartment has a living room, bedrooms, kitchen, bathroom etc.  Once you're in an apartment you don't know the difference!)

\section{Key goals}

\begin{description}
\item[Transparency:] the virtualised address space must look (to the process) as if it has an isolated physical address space.
\item[Efficiency:] virtualisation must not impose a (measurable) performance penalty. 
\item[Protection:] processes must be isolated from each other so that a malfunctioning or malicious process cannot alter contents of another. 
\end{description}


