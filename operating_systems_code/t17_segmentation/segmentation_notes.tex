\chapter{Segmentation}
\label{ch:segmentation}

% \begin{goals}
%   \begin{enumerate}
%   \item \textbf{Explain} the disadvantages of a sparse address space.
%   \item \textbf{Describe} the use of segmentation to better utilise physical memory.
%   \item \textbf{Convert} virtual to physical addresses in a segmented memory model.
%   \end{enumerate}
% \end{goals}

% \begin{reading}
%   \begin{enumerate}
%   \item
%     \citet[Chapter 16]{arpaci-dusseau:2015:operating}\\
%     \url{http://pages.cs.wisc.edu/~remzi/OSTEP/vm-segmentation.pdf}
%   \end{enumerate}
% \end{reading}

\section{Goal}

Memory is expensive, too expensive to waste on the unused space between the stack and heap.

It is quite possible for a running process that this unused space would be much greater than the utilised memory, a situation termed a \textbf{sparse address space}.

Can we break up this address space into segments that can be managed separately?

\section{Segmentation basics}

\citet[Chapter 16]{arpaci-dusseau:2015:operating} cites a number of early descriptions of segementation from the 1960's: \citet{greenfield:1962:fact}.

We define a \textbf{segment} to be a contiguous block of in-use memory.

Assuming that the address space consists of instructions, stack and heap, we divide the address space into three segments, one for each.
The three segments are then independently placed in the physical memory.

MMU must maintain separate base and bounds registers for each segment.


\section{Sample segmentation}

Consider a process with instructions, heap and stack.

\subsection{Instructions}

Assume that instructions start at $P^I_0$ and end at $P^I_1$, with corresponding virtual addresses $V^I_0$ to $V^I_1$.

Let $V_1$ denote a virtual address within the instruction portion of the address space.
With our usual model of code, stack and heap, this virtual address represents a simple offset from zero (in the virtual address space).

The corresponding physical address $P_1$ will be computed:
\begin{align}
  P_1 & = P^I_0 + V_1
\end{align}

Once this has been done, bounds checking will ensure that $P_1<P^I_1$.
Otherwise an exception will occur.

\subsection{Heap}

Let the heap be located in physical memory between $P^H_0$ and $P^H_1$, with corresponding virtual addresses $V^H_0$ to $V^H_1$.
The heap is slightly more complicated, because our memory model assumes that it follows the code.

Assume that we want to access a virtual address $V_2$ that falls within the heap segment.
We must account for the code when calculating the offset.

\begin{align}
  P_2 & = P^H_0 + V_2 - V^I_1
\end{align}


\section{Segment identification}

The hardware obviously needs to be able to identify the segment that any particular virtual address refers to.
There are two approaches:

\subsection{Explicit}

Recall that a memory address is merely a base-2 number.
The explicit approach uses bits in the memory address to identify the segment that an address belongs to.

For example, the upper two bits could be mapped as:
\begin{description}
\item[0 0 :] code
\item[0 1 :] heap
\item[1 0 :] stack
\end{description}

\subsection{Implicit}

Implicit segment identification takes advantage of what instruction a virtual address originates from:
\begin{description}
\item[Program counter] must be code.
\item[Stack or base pointer] must be the stack.
\item[Any other address] must be the heap.
\end{description}


\section{Stack}

The stack is more complex again, since it grows backwards!

Therefore, the hardware must have a bit that defines the direction that the segment grows towards.

The stack starts at $P^S_1$.
It runs \textbf{backwards} to $P^S_1$, so $P^S_1 \le P^S_0$.

Let the segment size, S, be:
\begin{align}
  S & = \left \lvert P^S_0 - P^S_1 \right \rvert \label{eq:segment-size}
\end{align}

Assuming that the stack begins at virtual address $V^S_0$, 
A virtual address, $V$, within the stack can be thought of as an offset into the stack address space of:
\begin{align}
  \Delta V^+ & = V - V^S_0 \label{eq:stack-offset}  
\end{align}

We then translate this to a negative offset by subtracting the max segment size from the offset:
\begin{align}
  \Delta V^- & = \Delta V^+ - S               
\end{align}

Then we add this negative offset to the base to arrive at the physical address, $P$, corresponding to the virtual address, $V$.
\begin{align}
  P & = P_0^S + \Delta V_-
\end{align}

\section{Code sharing}

Programs are normally not self-modifying.
If there are multiple identical running processes, we could assume that their code segments are identical.
Therefore, these can be shared.

Sharing requires that we have \textbf{protection bits} that define what processes are and are not allowed to do with memory segments:
\begin{description}
\item[Code (shared)] segments can be read and executed, but not written to by the process. 
\item[Stack and heap (per-process)] segments can be read and written to by the processes but are not executable.
\end{description}

An additional check (in addition to bounds-checking) is now required to ensure that the process uses memory only as allowed by the protection bits.
Misuse causes an exception that the operating system will be expected to handle.

\section{Granularity}

Segmentation can have a much more granular implementation than just code/stack/heap.

\section{Required OS support}

Operating system must ensure that segments are saved and restored on a context switch.
It must also manage the free space effectively to ensure that memory can be allocated when required.


%%% Local Variables:
%%% mode: latex
%%% TeX-master: "os_notes"
%%% End:

